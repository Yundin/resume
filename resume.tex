% !TEX TS-program = xelatex
% !TEX encoding = UTF-8 Unicode
% !TEX spellcheck = ru-RU

\documentclass[a4paper,12pt]{article}

\usepackage[a4paper, margin=2cm]{geometry}

\usepackage{fontspec}
\setmainfont{Arial}

\setlength{\parindent}{0em}
\setlength{\parskip}{1ex}

\usepackage[dvipsnames]{xcolor}
\usepackage{hyperref}
\hypersetup{
	unicode=true,
	pdftitle={Resume Yundin},
	pdfauthor={Yundin Vladislav},
	colorlinks=true,
    urlcolor=RoyalBlue
}

\pagenumbering{gobble}

\usepackage{graphicx}
\usepackage{wrapfig}

\emergencystretch=1.4em

\begin{document}
    {\huge Юндин Владислав Андреевич}

    \setlength\intextsep{0pt}
    \begin{wrapfigure}[5]{l}{0.31\textwidth}
        \includegraphics[width=0.3\textwidth]{avatar_square}
    \end{wrapfigure}

    \bigskip
    Контакты: 4yundin@gmail.com, \href{tel:+79999640461}{\color{black}+7 (999) 964-04-61}\par
    GitHub: github.com/Yundin
    
    Приложения: \href{https://play.google.com/store/apps/details?id=com.juntoteam.spotpet}{SpotPet}, \href{https://play.google.com/store/apps/details?id=com.juntoteam.keeppet}{KeepPet}, \href{https://play.google.com/store/apps/details?id=com.app.qavala1}{DaxApp}
    
    Город: Москва\par
    Дата рождения: 08.08.1998\par

    \vspace{7ex}
    \section*{О себе}

    3 года пишу под Android.

    Очень люблю vim без плагинов, IdeaVim, шорткаты. 

    Любимая среда для всего, кроме Android-разработки, --- терминал.

    Уважаю open source, поэтому Android.

    Примерно знаю, как всё работает снизу: на уровне электронов, транзисторов, триггеров, АЛУ, процессора, ассемблера. Знаю, как всё работает на уровне приложения. Пока копаю всё, что посередине.

    Не люблю Word, люблю \LaTeX.

    Прочитал git user-manual, теперь могу чинить сломанные ветки и красиво вести собственные.

    Знаю основные алгоритмы и структуры, но без хардкора.

    Английский на уровне Intermediate.

    Хочу научиться архитектуре.\\
    Хочу копнуть Gradle.

    \section*{С чем умею работать}% Технологии и уровень вледения

    \begin{itemize}
        \item ООП --- соблюдаю SOLID, использовать паттерны сложновато;
        \item Git --- понимаю, что происходит под капотом после команды;
        \item Java --- 3 года писал, читал <<Effective Java>>;
        \item Kotlin --- читал <<Kotlin in Action>>, написал два приложения;
        \item Custom view --- писал кастомные виджеты на основе FrameLayout c Canvas, кастомное ImageView, кастомный Drawable;
        \item Retrofit --- писал свой Call Adapter;
        \item Dagger --- использовал с одним компонентом и c AndroidInjection;
        \item RxJava --- использовал;
        \item Kotlin coroutenes --- использовал;
        \item Moxy --- использовал;
        \item Realm --- использовал;
        \item Google Maps API --- использовал кастомные маркеры, кластеризацию.
    \end{itemize}

    \section*{Ожидания от компании}

    Должность: Android-разработчик.\par
    Занятость: полная, офис.\par
    З/п: от 100 000 рублей net.\par
    Также:
    \begin{itemize}
        \item Поддержка обучения;
        \item Возможность влиять на продукт;
        \item Возможность консультации с более опытными коллегами;
        \item Питание;
        \item ДМС.
    \end{itemize}

    \section*{Опыт работы}

    \subsection*{DaxApp, октябрь 2019 --- декабрь 2019}

    https://daxapp.com\par
    Middle Android-разработчик\par
    Редизайн приложения, работа с графиками и кастомными виджетами.

    Много работал с \href{https://github.com/PhilJay/MPAndroidChart}{MPAndroidChart}, делал стобчатые, линейные, круговые диаграммы с небольшими кастомизациями. Писал кастомные виджеты на FrameLayout и немного Canvas для сложных теней. Вот \href{https://play.google.com/store/apps/details?id=com.juntoteam.spotpet}{результат}.
    
    \subsection*{Junto, май 2018 --- март 2019}
    
    https://juntoteam.com\par
    Junior Android-разработчик\par
    Разработка клиент-серверных приложений.\par
    
    Во время работы в компании в одиночку написал \href{https://play.google.com/store/apps/details?id=com.juntoteam.keeppet}{KeepPet} и, c небольшой помощью коллег, \href{https://play.google.com/store/apps/details?id=com.juntoteam.spotpet}{SpotPet}. 
    
    В KeepPet наиболее интересной задачей стал экран родословной питомца, который должен был отображать потомков и предков питомца, при этом поддерживать зум и перемещение. Для реализации я решил воспользоваться динамически наполняемым ConstraintLayout вместе с \href{https://github.com/natario1/ZoomLayout}{библиотекой}. В ходе тестирования в библиотеке был обнаружен баг с жестами, мой Pull Request с решением вошел в следующую её версию. Также была проделана большая работа по отображению на google картах  кастомных маркеров и их кластеризации.
    
    В SpotPet интересными были места, категории которых приходили с сервера и определяли наличие и порядок полей на экранах просмотра и создания конкретного места. То есть приложение знало только то, из чего может состоять категория. Реализовано было через несколько ViewStub и динамического раздувания контента.

    \section*{Другие проекты}

    \subsection*{Студенческая организация <<Бизнес в стиле .RU>>, сентябрь 2017 --- настоящее время}

    https://styleru.org\par
    Глава отдела Android-разработки\par
    Обучение студентов основам разработки, разработка клиент-серверных приложений.  

    \section*{Образование}

    \subsection*{Бакалавриат}

    Национальный исследовательский университет <<Высшая школа экономики>>\par
    Факультет: МИЭМ\par
    Направление: Информатика и вычислительная техника\par
    Специализация: Вычислительные системы и компьютерные сети\par
    Годы обучения: 2016 --- 2020
\end{document}
